\begin{frame}{Polynomial time}
  
  \redmath{P = \bigcup_k \text{TIME}(n^k)}
  
  \begin{description}
    \item[Decidable] languages are languages for which there is at least one Turing machine that halts in a finite number of state table lookups for each input.
    \item[P] is the set of languages that are decidable in polynomial time using a deterministic Turing machine.
    \item[Polynomial] means that for a length of input \( n \) the number of steps (state table lookups) is \( O(n^k) \) for some \( k \in \mathbb{N} \).
    \item[\( \text{TIME}(n^k) \)] is the set of languages decidable in \( O(n^k) \) steps.
  \end{description}

\end{frame}


\begin{frame}[fragile]{Recap on polynomials}
  
  \redmath{a_0 + a_1 x + a_2 x^2 + \cdots + a_m x^m \text{ where } a_i \in \mathbb{R}, m \in \mathbb{N}}
  
  \begin{description}
    \item[Polynomials] have a fixed number of operations (add, multiply) to perform for all values of \( x \).
    \item[Exponential] constrasts with polynomial, having a variable number of operations, e.g. \( 2^x \).
    \item[Functions] that are polynomial in $n$ are functions that take $n$ and plug it into a polynomial to give the result.
  \end{description}

  \begin{adjustbox}{max width={0.4\textwidth}, center}
    \begin{tikzpicture}
      \begin{axis}[xmin=0, domain=0:3, axis x line=bottom, axis y line=left, legend style={at={(0.4,0.4)},anchor=south}]
        \addplot[black] {pow(x,4) + 5*pow(x, 2) + 1};
        \legend{$x^4 + 5 x^2 + 1$};
      \end{axis}
    \end{tikzpicture}
  \end{adjustbox}

\end{frame}


\begin{frame}[fragile]{Polynomials are closed}
  
  \redmath{\mathbb{R}[x] = \{ a_0 + a_1 x + a_2 x^2 + \cdots + a_m x^m \mid a_i \in \mathbb{R}, m \in \mathbb{N} \}}
  
  The set of all polynomials is closed under the following operations:
  \begin{description}
    \item[Addition:] \( (x^4 + 1) + (5 x^3 + x) = x^4 + 5 x^3 + x + 1 \)
    \item[Multiplication:] \( (x^4 + 1) \times (5 x^3 + x) = 5 x^7 + x^5 + 5 x^3 + x \)
    \item[Composition:] \( ((5 x^3 + x))^4 + 1  = 625x^{12} + 500 x^{10} + 150x^8 + 20x^6 + x^4 \)
  \end{description}

  \red{Applying two polynomial time algorithms, one after the other, is still polynomial time.}

\end{frame}


\begin{frame}{Tractability}
  
  \begin{description}
    \item[Cobham's] thesis is that P represents the tractable problems.
    \item[Strictly] exponential time problems are considered intractable.
    \item[Careful:] would you consider an problem whose best known algorithm is $O(n^{100000})$ tractable?
  \end{description}

  \red{The exponential time hypothesis is that 3-SAT can't be solved in subexponential time. 3-SAT is NP-complete.}

\end{frame}