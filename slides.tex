\documentclass[dvipsnames,hidelinks,t]{beamer}

  % Enables the use of colour.
  \usepackage{xcolor}
  % Syntax high-lighting for code. Requires Python's pygments.
  \usepackage{minted}
  % Enables the use of umlauts and other accents.
  \usepackage[utf8]{inputenc}
  % Diagrams.
  \usepackage{tikz}
  % Settings for captions, such as sideways captions.
  \usepackage{caption}
  % Symbols for units, like degrees and ohms.
  \usepackage{gensymb}
  % Latin modern fonts - better looking than the defaults.
  \usepackage{lmodern}
  % Allows for columns spanning multiple rows in tables.
  \usepackage{multirow}
  % Better looking tables, including nicer borders.
  \usepackage{booktabs}
  % More math symbols.
  \usepackage{amssymb}
  % More math fonts, like mathbb.
  \usepackage{amsfonts}
  % More math layouts, equation arrays, etc.
  \usepackage{amsmath}
  % More theorem environments.
  \usepackage{amsthm}
  % More column formats for tables.
  \usepackage{array}
  % Adjust the sizes of box environments.
  \usepackage{adjustbox}
  % Better looking single quotes in verbatim and minted environments.
  \usepackage{upquote}
  % Better blank space decisions.
  \usepackage{xspace}
  % Better looking tikz trees.
  \usepackage{forest}
  % URLs.
  \usepackage{hyperref}
  % For plotting.
  \usepackage{pgfplots}
  
  % Various tikz libraries.
  % For drawing mind maps.
  \usetikzlibrary{mindmap}
  % For adding shadows.
  \usetikzlibrary{shadows}
  % Extra arrows tips.
  \usetikzlibrary{arrows.meta}
  % Old arrows.
  \usetikzlibrary{arrows}
  % Automata.
  \usetikzlibrary{automata}
  % For more positioning options.
  \usetikzlibrary{positioning}
  % Creating chains of nodes on a line.
  \usetikzlibrary{chains}
  % Fitting node to contain set of coordinates.
  \usetikzlibrary{fit}
  % Extra shapes for drawing.
  \usetikzlibrary{shapes}
  % For markings on paths.
  \usetikzlibrary{decorations.markings}
  % For advanced calculations.
  \usetikzlibrary{calc}
  
  % GMIT colours.
  \definecolor{gmitblue}{RGB}{20,134,225}
  \definecolor{gmitred}{RGB}{220,20,60}
  \definecolor{gmitgrey}{RGB}{67,67,67}
  
  % Change some style options.
  \usetheme{metropolis}
  % Tell minted to use the following colour scheme. 
  \usemintedstyle{manni}
  % Remove some of the vertical space after the title.
  % \addtobeamertemplate{frametitle}{}{\vspace{-3mm}}
  % Change the default theme colours.
  \setbeamercolor{normal text}{fg=darkgray, bg=white}
  \setbeamercolor{alerted text}{fg=gmitred, bg=white}
  \setbeamercolor{example text}{fg=gmitblue, bg=white}
  \setbeamercolor{frametitle}{fg=gmitblue, bg=white}
  \setbeamercolor*{item}{fg=gmitblue}
  % Use a better math mode font.
  \usefonttheme[onlymath]{serif}
  % Don't display section pages.
  \metroset{sectionpage=none}
  % Change the default itemize bullets.
  \setbeamertemplate{itemize item}{\color{gray}--}
  % Change the position of left aligned math.
  %\setlength{\mathindent}{7mm}

  % An environment for displaying math in red, without lots of vertical space.
  \newcommand{\redmath}[1]{\vspace{-3mm} {\begin{center} \color{gmitred} \( #1 \) \end{center}} \vspace{-2mm}}
  \newcommand{\red}[1]{\vspace{-3mm} {\begin{center} \color{gmitred} #1 \end{center}} \vspace{-2mm}}

  % For displaying a blank character.
  \newcommand{\bl}{\underline{\hspace{2mm}}}

  % \citeurl can be used to a clickable short url to a slide as a reference.
  \renewcommand\footnoterule{}
  \newcommand{\citeurl}[1]{\let\thefootnote\relax\footnotetext{\tiny \textcolor{gmitgrey}{\href{http://#1}{#1}}}}
  \newcommand{\citeeg}[1]{\let\thefootnote\relax\footnotetext{\tiny \textcolor{gmitgrey}{#1}}}
  
  % Prevent minted from showing errors.
  \makeatletter
  \expandafter\def\csname PYGdefault@tok@err\endcsname{\def\PYGdefault@bc##1{{\strut ##1}}}
  \makeatother
  
  \begin{document}
    \title{Polynomial time}
    \subtitle{}
    \author{ian.mcloughlin@gmit.ie}
    \date{}
  
    \begin{frame}
      \titlepage
    \end{frame}
  
    \begin{frame}{Polynomial time}
  
  \redmath{P = \bigcup_k \text{TIME}(n^k)}
  
  \begin{description}
    \item[Decidable] languages are languages for which there is at least one Turing machine that halts in a finite number of state table lookups for each input.
    \item[P] is the set of languages that are decidable in polynomial time using a deterministic Turing machine.
    \item[Polynomial] means that for a length of input \( n \) the number of steps (state table lookups) is \( O(n^k) \) for some \( k \in \mathbb{N} \).
    \item[\( \text{TIME}(n^k) \)] is the set of languages decidable in \( O(n^k) \) steps.
  \end{description}

\end{frame}


\begin{frame}[fragile]{Recap on polynomials}
  
  \redmath{a_0 + a_1 x + a_2 x^2 + \cdots + a_m x^m \text{ where } a_i \in \mathbb{R}, m \in \mathbb{N}}
  
  \begin{description}
    \item[Polynomials] have a fixed number of operations (add, multiply) to perform for all values of \( x \).
    \item[Exponential] constrasts with polynomial, having a variable number of operations, e.g. \( 2^x \).
    \item[Functions] that are polynomial in $n$ are functions that take $n$ and plug it into a polynomial to give the result.
  \end{description}

  \begin{adjustbox}{max width={0.4\textwidth}, center}
    \begin{tikzpicture}
      \begin{axis}[xmin=0, domain=0:3, axis x line=bottom, axis y line=left, legend style={at={(0.4,0.4)},anchor=south}]
        \addplot[black] {pow(x,4) + 5*pow(x, 2) + 1};
        \legend{$x^4 + 5 x^2 + 1$};
      \end{axis}
    \end{tikzpicture}
  \end{adjustbox}

\end{frame}


\begin{frame}[fragile]{Polynomials are closed}
  
  \redmath{\mathbb{R}[x] = \{ a_0 + a_1 x + a_2 x^2 + \cdots + a_m x^m \mid a_i \in \mathbb{R}, m \in \mathbb{N} \}}
  
  The set of all polynomials is closed under the following operations:
  \begin{description}
    \item[Addition:] \( (x^4 + 1) + (5 x^3 + x) = x^4 + 5 x^3 + x + 1 \)
    \item[Multiplication:] \( (x^4 + 1) \times (5 x^3 + x) = 5 x^7 + x^5 + 5 x^3 + x \)
    \item[Composition:] \( ((5 x^3 + x))^4 + 1  = 625x^{12} + 500 x^{10} + 150x^8 + 20x^6 + x^4 \)
  \end{description}

  \red{Applying two polynomial time algorithms, one after the other, is still polynomial time.}

\end{frame}


\begin{frame}{Tractability}
  
  \begin{description}
    \item[Cobham's] thesis is that P represents the tractable problems.
    \item[Strictly] exponential time problems are considered intractable.
    \item[Careful:] would you consider an problem whose best known algorithm is $O(n^{100000})$ tractable?
  \end{description}

  \red{The exponential time hypothesis is that 3-SAT can't be solved in subexponential time. 3-SAT is NP-complete.}

\end{frame} 
  \end{document}